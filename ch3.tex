\chapter{Add two vectors in C++ and R}

\section{C++ code}
The C++ example in which I write a function to add two vectors together uses a traditional loop to do element by element addition

\lstset{language=C++, showspaces=false, showstringspaces=false}
\lstinputlisting{code/ex3.cpp}

\section{R code}
Equivalent code in R. Notice how there is no need to define a function to add two vectors. Simply use the + operator.

\lstset{language=R, showspaces=false, showstringspaces=false}
\lstinputlisting{code/ex3.R}

\section{Rcpp code}
TODO write ex3\_rcpp.R to add two vectors together
%\lstset{language=R, showspaces=false, showstringspaces=false}
%\lstinputlisting{code/ex2_rcpp.R}

\section{Output}

Output of ex3.cpp
\begin{verbatim}
$ make ex3
$ ./ex3
21 23 25 27 29 31 33 35 37 39
\end{verbatim}
Output of ex3.R
\begin{verbatim}
$ Rscript ex3.R
21 23 25 27 29 31 33 35 37 39
\end{verbatim}