\chapter{Basic Arithmetic Operations on Vectors in C++ and R}
The first thing to notice is that with functions in C++, the return type and argument type have to be declared in the function. Also note functions can have the same name with different arguments. In R, if x and y are vectors we can add the vectors with x + y. We can't do this in C++, at least not without overloading the + operator. Rcpp sugar makes this possible with a little bit of what seems like "magic". Rcpp sugar will be covered later.

\section{Addition}
Simple program to add two vectors and add a scalar to a vector

\lstset{language=C++}
\lstinputlisting{code/addition.cpp}

\lstset{language=R}
\lstinputlisting{code/addition.R}

\section{Subtraction}
Simple program to subtract two vectors and subtract a vector by a scalar

\lstset{language=C++}
\lstinputlisting{code/subtraction.cpp}

\lstset{language=R}
\lstinputlisting{code/subtraction.R}

\section{Multiplication}
Simple program to multiply two vectors and multiply a vector by a scalar

\lstset{language=C++}
\lstinputlisting{code/multiplication.cpp}

\lstset{language=R}
\lstinputlisting{code/multiplication.R}

\section{Division}
Simple program to divide two vectors and divide a vector by a scalar

\lstset{language=C++}
\lstinputlisting{code/division.cpp}

\lstset{language=R}
\lstinputlisting{code/division.R}